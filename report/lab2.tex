%%%%%%%%%%%%%%%%%%%%%%%%%%%%%%%%%%%%%%%%%
% Short Sectioned Assignment
% LaTeX Template
% Version 1.0 (5/5/12)
%
% This template has been downloaded from:
% http://www.LaTeXTemplates.com
%
% Original author:
% Frits Wenneker (http://www.howtotex.com)
%
% License:
% CC BY-NC-SA 3.0 (http://creativecommons.org/licenses/by-nc-sa/3.0/)
%
%%%%%%%%%%%%%%%%%%%%%%%%%%%%%%%%%%%%%%%%%

%------------------------------------------------------------------------------
%	PACKAGES AND OTHER DOCUMENT CONFIGURATIONS
%------------------------------------------------------------------------------

\documentclass[paper=a4, fontsize=11pt]{scrartcl} % A4 paper and 11pt font size

\usepackage[T1]{fontenc} % Use 8-bit encoding that has 256 glyphs
%\usepackage{fourier} % Use the Adobe Utopia font for the document - comment this line to return to the LaTeX default
\usepackage[english]{babel} % English language/hyphenation
\usepackage{amsmath,amsfonts,amsthm} % Math packages

\usepackage[utf8]{inputenc} % Needed to support swedish "åäö" chars
\usepackage{titling} % Used to re-style maketitle
\usepackage{enumerate}
\usepackage{lipsum} % Used for inserting dummy 'Lorem ipsum' text into the template

\usepackage{hyperref}
\usepackage{url}

\usepackage{sectsty} % Allows customizing section commands
\allsectionsfont{\normalfont} % Make all sections centered, the default font and small caps

\usepackage{fancyhdr} % Custom headers and footers
\pagestyle{fancyplain} % Makes all pages in the document conform to the custom headers and footers
\fancyhead{} % No page header - if you want one, create it in the same way as the footers below
\fancyfoot[L]{} % Empty left footer
\fancyfoot[C]{} % Empty center footer
\fancyfoot[R]{\thepage} % Page numbering for right footer
\renewcommand{\headrulewidth}{0pt} % Remove header underlines
\renewcommand{\footrulewidth}{0pt} % Remove footer underlines
\setlength{\headheight}{13.6pt} % Customize the height of the header

\numberwithin{equation}{section} % Number equations within sections (i.e. 1.1, 1.2, 2.1, 2.2 instead of 1, 2, 3, 4)
\numberwithin{figure}{section} % Number figures within sections (i.e. 1.1, 1.2, 2.1, 2.2 instead of 1, 2, 3, 4)
\numberwithin{table}{section} % Number tables within sections (i.e. 1.1, 1.2, 2.1, 2.2 instead of 1, 2, 3, 4)

\setlength\parindent{0pt} % Removes all indentation from paragraphs - comment this line for an assignment with lots of text

\usepackage{fancyvrb}
\DefineShortVerb{\|}


\posttitle{\par\end{center}} % Remove space between author and title
%----------------------------------------------------------------------------------------
% TITLE SECTION
%----------------------------------------------------------------------------------------

\title{ 
\huge Laboration 2 \\ Small-Shell v2.51 för UNIX \\ % The assignment title
\vspace{10pt}
\normalfont \normalsize 
\textsc{ID2200 - Operating Systems } \\ [25pt] %
}

\author{Gustaf Lindstedt \\ glindste@kth.se \\ 910301-2135 \and Martin Runelöv \\ mrunelov@kth.se \\ 900330-5738}

\date{\vspace{8pt}\normalsize\today} % Today's date or a custom date

\begin{document}

\maketitle

\section{Inledning}

% OBS! Se till så att filerna är läsbara för assistenterna (se 3.4 ).
Koden finns tillgänglig i följande sökväg:\\
|/afs/ict.kth.se/home/m/r/mrunelov/os/lab2/shell.c|.\\

samt i appendix.


% Verksamhetsberättelse och synpunkter på laborationen. Beskriv arbetets utveckling. Hade du problem med verktygen,
% kompilatorn m.fl.? Hur lång tid har arbetet tagit? Skriv dina förslag till förändringar, idéer etc. Tyck fritt! 
% Vi är angelägna om att få respons, så att vi kan förbättra till nästa år.

% Uppskattning av tidsåtgång och eventuella kommentarer kring laborationen
\subsection{Verksamhetsberättelse}
Total tidsåtgång: 10 timmar. I princip allt arbete utfördes gemensamt.\\

Vi valde att delvis följa rekommendationerna i labbpeket.
Vi började med att implementera en prompt som kunde läsa in ett kommando och dess argument i en |char**|.
Sedan fortsatte vi med implementering av förgrundsprocesser med hjälp av |fork| och |execvp|.
Det var inte helt enkelt att implementera statistikutskrifter för dessa då tidtagning visade sig vara lite problematiskt.
Därefter implementerade vi de inbyggda kommandona |cd| och |exit|.
Sedan la vi till bakgrundsprocesser som kontrolleras med pollning.
Detta gick oväntat smidigt och fungerade i princip direkt.
Det svåraste med bakgrundsprocesser visade sig vara att parsa |&|-symbolen,
eftersom den antingen kan stå för sig själv eller vara avslutande |char| i sista argumentet.
Något som visade sig svårare än vi trodde var att se till att vår shell ignorerade |SIGINT|.
Detta var på grund av att vår loop körde tills vi fick |NULL| som resultat från |fgets()|.
Men det tog ett tag för oss att inse att |fgets()| även returnerar NULL vid andra fel än när EOF påträffats, till exempel när den avbröts av en signal.



\subsection{Synpunkter}
Ännu en gång var dokumentationen som fanns tillgänglig väldigt bra, framförallt labbpeket.

Det är en rolig labb där man får repetera en del stränghantering i C 
och befästa sina kunskaper om processer, signaler och allmän process-kommunikation.


\section{Förberedelsefrågor}
\begin{enumerate}[1)]

\item 
\emph{Motivera varför det ofta är bra att exekvera kommandon i en separat process:}

-- Om man skapar nya kommandon i separata processer kan man ha kvar en 
övervakande process som t.ex. hanterar signaler från dessa barnprocesser. 
Med detta upplägg kan man bland annat hantera scenarion när kommandon 
leder till fel. Om man inte skapar en ny process kan då hela programmet krascha.

\item
\emph{Vad händer om man inte i kommandotolken exekverar wait() för en 
barn-process som avslutas?}

-- Det ligger kvar onödig information i processtabellen, 
eftersom operativsystemet väntar på att någon ska kontrollera status för processen. 
Processen är då en så kallad zombieprocess.

\item
\emph{Hur skall man utläsa SIGSEGV?}

''Segmentation violation'' (SIG = signal). 
Vanligare namn är ''segmentation fault'' (eller ''access violation'').
Felet indikerar att ett försök till ogiltig minnesläsning skedde.

\item
\emph{Varför kan man inte blockera SIGKILL?} % TODO: Se över svaret!

-- För att man ska kunna tvinga processer att pausa (|SIGSTOP|) och avsluta.

% Detta finns dokumenterat i man-filen för sigaction:
% \begin{verbatim}
% ERRORS
%      The sigaction() system call will fail and no new signal handler will be
%      installed if one of the following occurs:

%      [EINVAL]           An attempt is made to ignore or supply a handler for
%                         SIGKILL or SIGSTOP.
% \end{verbatim}

\item
\emph{Hur skall man utläsa deklarationen} |void (*disp)(int)|?

-- Man deklarerar en funktionspekare, |disp|, som pekar på en funktion som 
tar en int-parameter och returnerar void.

\item
\emph{Vilket systemanrop använder sigset(3C) troligtvis för att 
installera en signalhanterare?} % TODO: Se över.

-- sigaction(2) eller signal(2).

sigaction verkar vara det man använder nuförtiden då signal inte är lika portabel.

\item
\emph{Hur gör man för att din kommandotolk inte skall termineras 
då en förgrundsprocess i den termineras med <Ctrl-c>?}

-- Genom att registrera en signal handler som hanterar |SIGINT|. 
Man borde dock ta bort denna handler efter att förgrundsprocessen har avslutats
då det inte är särskilt användarvänligt att förhindra |SIGINT| i huvudprogrammet.

\item
\emph{Studera körningsexemplet nedan och förklara varför man inte har bytt 
''working directory'' till /home/ds/robertr när man avslutat miniShell:et?}

-- När processen för miniShell skapas får den en kopia på environment-variabler. Processen har alltså ett eget, lokalt 
working directory. När programmet avslutas ser vi återigen den tidigare processens working directory.

\end{enumerate}

\section{Problembeskrivning}

Uppgiften går ut på att skriva en enkel kommandotolk 
som hanterar både förgrunds- och bakgrundsprocesser.
Utöver att delegera anrop till |exec| ska två inbyggda kommandon,
 |cd| (change directory), och |exit| (avsluta) implementeras.

När något icke-inbyggt kommando ges startas en ny process, och |pid| för denna process skrivs ut.
Det ska även skrivas ut ett meddelande då processer avslutas, och om processen är en förgrundsprocess ska dess totala exekveringstid skrivas ut.


\section{Programbeskrivning}

Se appendix för källkod samt exempelkörningar.\\

Programmet börjar med att registrera en signal handler för SIGINT för huvudprocessen.
Sedan startas en evig loop som kontinuerligt läser in kommandon, med argument, och exekverar dessa.
Kommandotypen kontrolleras i följande ordning:
\begin{enumerate}
\item{Inbyggt kommando}
\item{Förgrundskommando}
\item{Bakgrundskommando}
\end{enumerate}

Om kommandot är ett inbyggt kommando exekveras det och loopen fortsätter.\\
Om kommandot ska köras i förgrunden skapas en child-process som exekverar kommandot
och parent-processen väntar på att child-processen ska avslutas innan loopen fortsätter.\\
Om kommandot är ett bakgrundskommando exekveras kommandot i en child-process och loopen fortsätter.\\

Efter att kommandot exekverats pollas det med ett icke-blockerande |waitpid|-anrop som kontrollerar ifall
någon bakgrundsprocess har avslutats. Detta görs även då loopen avbryts för att rensa bort zombie-processer innnan programmet avslutas.

% Nånting om zombieprocesser. De existerar vid SIGKILL t.ex.



\newpage
\section*{Appendix}
\subsection*{Exempelkörning}
%------------------------------------------------------------------------------

\end{document}
