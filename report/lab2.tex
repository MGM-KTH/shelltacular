%%%%%%%%%%%%%%%%%%%%%%%%%%%%%%%%%%%%%%%%%
% Short Sectioned Assignment
% LaTeX Template
% Version 1.0 (5/5/12)
%
% This template has been downloaded from:
% http://www.LaTeXTemplates.com
%
% Original author:
% Frits Wenneker (http://www.howtotex.com)
%
% License:
% CC BY-NC-SA 3.0 (http://creativecommons.org/licenses/by-nc-sa/3.0/)
%
%%%%%%%%%%%%%%%%%%%%%%%%%%%%%%%%%%%%%%%%%

%------------------------------------------------------------------------------
%	PACKAGES AND OTHER DOCUMENT CONFIGURATIONS
%------------------------------------------------------------------------------

\documentclass[paper=a4, fontsize=11pt]{scrartcl} % A4 paper and 11pt font size

\usepackage[T1]{fontenc} % Use 8-bit encoding that has 256 glyphs
%\usepackage{fourier} % Use the Adobe Utopia font for the document - comment this line to return to the LaTeX default
\usepackage[english]{babel} % English language/hyphenation
\usepackage{amsmath,amsfonts,amsthm} % Math packages

\usepackage[utf8]{inputenc} % Needed to support swedish "åäö" chars
\usepackage{titling} % Used to re-style maketitle
\usepackage{enumerate}
\usepackage{lipsum} % Used for inserting dummy 'Lorem ipsum' text into the template

\usepackage{hyperref}
\usepackage{url}
\usepackage{graphicx}
\usepackage[left=3.5cm, right=3.5cm]{geometry} % margins for title page. changed below.

\usepackage{sectsty} % Allows customizing section commands
\allsectionsfont{\normalfont} % Make all sections centered, the default font and small caps

\usepackage{fancyhdr} % Custom headers and footers
\pagestyle{fancyplain} % Makes all pages in the document conform to the custom headers and footers
\fancyhead{} % No page header - if you want one, create it in the same way as the footers below
\fancyfoot[L]{} % Empty left footer
\fancyfoot[C]{} % Empty center footer
\fancyfoot[R]{\thepage} % Page numbering for right footer
\renewcommand{\headrulewidth}{0pt} % Remove header underlines
\renewcommand{\footrulewidth}{0pt} % Remove footer underlines
\setlength{\headheight}{13.6pt} % Customize the height of the header

\numberwithin{equation}{section} % Number equations within sections (i.e. 1.1, 1.2, 2.1, 2.2 instead of 1, 2, 3, 4)
\numberwithin{figure}{section} % Number figures within sections (i.e. 1.1, 1.2, 2.1, 2.2 instead of 1, 2, 3, 4)
\numberwithin{table}{section} % Number tables within sections (i.e. 1.1, 1.2, 2.1, 2.2 instead of 1, 2, 3, 4)

\setlength\parindent{0pt} % Removes all indentation from paragraphs - comment this line for an assignment with lots of text

\usepackage{fancyvrb}
\DefineShortVerb{\|}


\posttitle{\par\end{center}} % Remove space between author and title
%----------------------------------------------------------------------------------------
% TITLE SECTION
%----------------------------------------------------------------------------------------

\title{ 
\huge Laboration 2 \\ Small-Shell v2.51 för UNIX \\ % The assignment title
\vspace{10pt}
\normalfont \normalsize 
\textsc{ID2200 - Operating Systems } \\ [25pt] %
}

\author{Gustaf Lindstedt \\ glindste@kth.se \\ 910301-2135 \and Martin Runelöv \\ mrunelov@kth.se \\ 900330-5738}

\date{\vspace{8pt}\normalsize\today} % Today's date or a custom date

\begin{document}

\maketitle

\section{Inledning}

% OBS! Se till så att filerna är läsbara för assistenterna (se 3.4 ).
Koden finns tillgänglig i följande sökväg:\\
|/afs/ict.kth.se/home/m/r/mrunelov/os/lab2/shell.c|.\\

samt i appendix.


% Verksamhetsberättelse och synpunkter på laborationen. Beskriv arbetets utveckling. Hade du problem med verktygen,
% kompilatorn m.fl.? Hur lång tid har arbetet tagit? Skriv dina förslag till förändringar, idéer etc. Tyck fritt! 
% Vi är angelägna om att få respons, så att vi kan förbättra till nästa år.

% Uppskattning av tidsåtgång och eventuella kommentarer kring laborationen
\subsection{Verksamhetsberättelse}
Total tidsåtgång: 10 timmar. I princip allt arbete utfördes gemensamt.\\

Vi valde att delvis följa rekommendationerna i labbpeket.
Vi började med att implementera en prompt som kunde läsa in ett kommando och dess argument i en |char**|.
Sedan fortsatte vi med implementering av förgrundsprocesser med hjälp av |fork| och |execvp|.
Det var inte helt enkelt att implementera statistikutskrifter för dessa då tidtagning visade sig vara lite problematiskt.
Därefter implementerade vi de inbyggda kommandona |cd| och |exit|.
Sedan la vi till bakgrundsprocesser som kontrolleras med pollning.
Detta gick oväntat smidigt och fungerade i princip direkt.
Det svåraste med bakgrundsprocesser visade sig vara att parsa |&|-symbolen,
eftersom den antingen kan stå för sig själv eller vara avslutande |char| i sista argumentet.
Något som visade sig svårare än vi trodde var att se till att vår shell ignorerade |SIGINT|.
Detta var på grund av att vår loop körde tills vi fick |NULL| som resultat från |fgets()|.
Men det tog ett tag för oss att inse att |fgets()| även returnerar NULL vid andra fel än när EOF påträffats, till exempel när den avbröts av en signal.


\subsection{Synpunkter}
Ännu en gång var dokumentationen som fanns tillgänglig väldigt bra, framförallt labbpeket.\\

Det är en rolig labb där man får repetera en del stränghantering i C 
och befästa sina kunskaper om processer, signaler och allmän process-kommunikation.\\

it.kth.se-servrarna är mycket långsammare än de vi är vana vid på CSC.
Vi har ingen aning om varför, men det kanske kan åtgärdas.




\section{Förberedelsefrågor}
\begin{enumerate}[1)]

\item 
\emph{Motivera varför det ofta är bra att exekvera kommandon i en separat process:}

-- Om man skapar nya kommandon i separata processer kan man ha kvar en 
övervakande process som t.ex. hanterar signaler från dessa barnprocesser. 
Med detta upplägg kan man bland annat hantera scenarion när kommandon 
leder till fel. Om man inte skapar en ny process kan då hela programmet krascha.

\item
\emph{Vad händer om man inte i kommandotolken exekverar wait() för en 
barn-process som avslutas?}

-- Det ligger kvar onödig information i processtabellen, 
eftersom operativsystemet väntar på att någon ska kontrollera status för processen. 
Processen är då en så kallad zombieprocess.

\item
\emph{Hur skall man utläsa SIGSEGV?}

''Segmentation violation'' (SIG = signal). 
Vanligare namn är ''segmentation fault'' (eller ''access violation'').
Felet indikerar att ett försök till ogiltig minnesläsning skedde.

\item
\emph{Varför kan man inte blockera SIGKILL?} % TODO: Se över svaret!

-- För att man (särskilt operativsystemet) ska kunna tvinga processer att pausa (|SIGSTOP|) och avsluta.

% Detta finns dokumenterat i man-filen för sigaction:
% \begin{verbatim}
% ERRORS
%      The sigaction() system call will fail and no new signal handler will be
%      installed if one of the following occurs:

%      [EINVAL]           An attempt is made to ignore or supply a handler for
%                         SIGKILL or SIGSTOP.
% \end{verbatim}

\item
\emph{Hur skall man utläsa deklarationen} |void (*disp)(int)|?

-- Man deklarerar en funktionspekare, |disp|, som pekar på en funktion som 
tar en int-parameter och returnerar void.

\item
\emph{Vilket systemanrop använder sigset(3C) troligtvis för att 
installera en signalhanterare?} % TODO: Se över.

-- sigaction(2) eller signal(2). sigaction verkar vara det man använder nuförtiden då signal inte är lika portabel.

\item
\emph{Hur gör man för att din kommandotolk inte skall termineras 
då en förgrundsprocess i den termineras med <Ctrl-c>?}

-- Genom att registrera en signal handler som hanterar |SIGINT|.
Ofta vill man dock ta bort denna handler efter att förgrundsprocessen har avslutats
då det inte är särskilt användarvänligt att förhindra |SIGINT| i huvudprogrammet.

\item
\emph{Studera körningsexemplet nedan och förklara varför man inte har bytt 
''working directory'' till /home/ds/robertr när man avslutat miniShell:et?}

-- När processen för miniShell skapas får den en kopia på environment-variabler. Processen har alltså ett eget, lokalt, working directory. När programmet avslutas ser vi återigen den tidigare processens working directory.

\end{enumerate}

\section{Problembeskrivning}

Uppgiften går ut på att skriva en enkel kommandotolk 
som hanterar både förgrunds- och bakgrundsprocesser.
Utöver att delegera anrop till |exec| ska två inbyggda kommandon,
 |cd| (change directory), och |exit| implementeras.

När något icke-inbyggt kommando ges startas en ny process, och |pid| för denna process skrivs ut.
Det ska även skrivas ut ett meddelande då processer avslutas, och om processen var en förgrundsprocess ska dess totala exekveringstid skrivas ut.


\section{Programbeskrivning}

Se appendix för exempelkörningar.\\

Programmet börjar med att registrera en signal handler för SIGINT för huvudprocessen.
Sedan startas en evig loop som kontinuerligt läser in kommandon, med argument, och exekverar dessa.
Kommandotypen kontrolleras i följande ordning:
\begin{enumerate}
\item{Inbyggt kommando}
\item{Förgrundskommando}
\item{Bakgrundskommando}
\end{enumerate}

Om kommandot är ett inbyggt kommando exekveras det och loopen fortsätter.\\
Om kommandot ska köras i förgrunden skapas en child-process som exekverar kommandot
och parent-processen väntar på att child-processen ska avslutas innan loopen fortsätter.\\
Om kommandot är ett bakgrundskommando exekveras kommandot i en child-process och loopen fortsätter.\\

Efter att kommandot exekverats sker pollning med ett icke-blockerande |waitpid|-anrop som kontrollerar ifall
någon bakgrundsprocess har avslutats. När loopen avbryts och programmet ska avslutas registreras en signal handler som automatiskt rensar bort zombieproceser. Efter detta körs en sista pollning för att undvika ett race condition där child-processer dog medan signal handlern registrerades. % TODO: Ska vi ha det så här? Ska vi isf beskriva det såhär?


%------------------------------------------------------------------------------
\newpage
\section*{Appendix}
\subsection*{Exempelkörning, Mac OS X (se beskrivning på nästa sida):}
%\newgeometry{left=0.5cm} % Set LR margins for the rest of the document
\centerline{\includegraphics[scale=0.9]{"test3"}}
\newpage
\subsection*{Flöde i ovanstående exempelkörning:}
\begin{align*}
&\text{Start förgrundsprocess (ls)} \rightarrow \text{Avslut förgrundsprocess (ls)} 
\\\\&\text{Start bakgrundsprocess (ls)} \rightarrow \text{Avslut bakgrundsprocess (ls)} 
\\\\&\text{Start bakgrundsprocess (cat)} \rightarrow \text{Start förgrundsprocess (kill)} 
\\& \rightarrow \text{Avslut förgrundsprocess (kill)} \rightarrow \text{Avslut bakgrundsprocess (cat)}
\\\\&\text{Start bakgrundsprocess (tick.sh)} \rightarrow \text{Avslut bakgrundsprocess (tick.sh)}  
\\\\&\text{Start bakgrundsprocess (eternal.sh)} \rightarrow \text{Start bakgrundsprocess (eternal.sh)} 
\\&\rightarrow \text{Start bakgrundsprocess (tick.sh)} \rightarrow \text{Avslut bakgrundsprocess (tick.sh)} 
\\&\rightarrow\text{Start förgrundsprocess (kill)} \rightarrow \text{Avslut förgrundsprocess (kill)} 
\\&\rightarrow\text{Start förgrundsprocess (kill)} \rightarrow \text{Avslut förgrundsprocess (kill)} 
\\&\rightarrow\text{Avslut bakgrundsprocess (eternal.sh)} \rightarrow \text{Avslut bakgrundsprocess (eternal.sh)} 
\end{align*}

\newpage
\subsection*{Exempelkörning, Ubuntu}
\centerline{\includegraphics[scale=0.8]{"shelltacular"}}

\newpage

\begin{verbatim}
/*
 * shelltacular - command interpreter (shell)
 *
 * SYNTAX
 *     shelltacular - start the shell
 *     BUILT-IN COMMANDS
 *         cd - change working directory
 *         exit - exit the shell
 *
 *     <cmd>     - run command <cmd> as a foreground process
 *     <cmd> &   - run command <cmd> as a background process
 *
 * DESCRIPTION
 *     shelltacular is a minimalistic shell with two built-in commands,
 *     cd and exit. All other commands will be executed using exec.
 *     The shell supports foreground and background processes.
 *     By default, a SIGINT signal handler is registered for all processes.
 *
 * OPTIONS
 *     N/A
 *
 * SEE ALSO
 *     bash(1), wait(2), exec(3), fork(2), signal(7), getenv(3), gettimeofday(2)
 *
 * AUTHORS
 *      Gustaf Lindstedt (glindste@kth.se)
 *      Martin Runelöv (mrunelov@kth.se)
 */

#define _POSIX_C_SOURCE 200112L

#include <stdio.h>
#include <stdlib.h>
#include <signal.h>
#include <string.h>
#include <unistd.h>
#include <sys/time.h>
#include <sys/types.h>

#ifdef __APPLE__
#include <sys/wait.h>
#else
#include <wait.h>
#endif

#define PROCESS_TYPE_FOREGROUND 1
#define PROCESS_TYPE_BACKGROUND 2
#define BUFSIZE 70
#define ARGSIZE 7 /* command included in ARGSIZE, last arg reserved for NULL */
#define PROMPT "$ "

/* PERTY COOLORS */
#define C_FG_BLACK   "\x1b[30m"
#define C_FG_RED     "\x1b[31m"
#define C_FG_GREEN   "\x1b[32m"
#define C_FG_YELLOW  "\x1b[33m"
#define C_FG_BLUE    "\x1b[34m"
#define C_FG_MAGENTA "\x1b[35m"
#define C_FG_CYAN    "\x1b[36m"
#define C_FG_WHITE   "\x1b[37m"
#define C_FG_RESET   "\x1b[39m"
#define BOLD_ON      "\x1b[1m"
#define BLINK_ON     "\x1b[5m"
#define ALL_OFF      "\x1b[0m"

void loop();
void newline();
void prompt();
void clearargs(char **args);
void printargs(char **args);
int getargs(char *buffer, char **args);
void spawn_command(char **args, int process_type);
void change_dir(char *directory);
void spawn_foreground_process(char **args);
void spawn_background_process(char **args);
void poll_background_processes();
void timeval_diff(struct timeval *diff, struct timeval *tv1, struct timeval *tv2);
void print_color(const char *string, int fgc, int bgc, int attr);

int NUM_BACKGROUND_PROCESSES = 0;

/* 
 * Calculates the difference (tv1-tv2) between two timevals. 
 *
 * based on source from:
 * http://www.gnu.org/software/libc/manual/html_node/Elapsed-Time.html 
 */
void timeval_diff(struct timeval *diff, struct timeval *tv1, struct timeval *tv2) {
    /* Perform the carry for the later subtraction by updating y. */
    if (tv1->tv_usec < tv2->tv_usec) {
        long int nsec = (tv2->tv_usec - tv1->tv_usec) / 1000000 + 1;
        tv2->tv_usec -= 1000000 * nsec;
        tv2->tv_sec += nsec;
    }
    if (tv1->tv_usec - tv2->tv_usec > 1000000) {
        long int nsec = (tv1->tv_usec - tv2->tv_usec) / 1000000;
        tv2->tv_usec += 1000000 * nsec;
        tv2->tv_sec -= nsec;
    }
     
    /* Compute the difference. tv_usec is positive. */
       diff->tv_sec = tv1->tv_sec - tv2->tv_sec;
       diff->tv_usec = tv1->tv_usec - tv2->tv_usec;
       return;
}

/*
 * Register a signal handler
 */
void register_sighandler( int signal_code, void (*handler) (int sig) )
{
    int retval;
    struct sigaction signal_parameters;

    signal_parameters.sa_handler = handler;
    sigemptyset(&signal_parameters.sa_mask);
    signal_parameters.sa_flags = 0;

    retval = sigaction(signal_code, &signal_parameters, NULL);

    if(-1 == retval) {
        perror("sigaction() failed");
        exit(1);
    }
}

/*
 * Handler for SIGINT
 */
void sigint_handler(int sig)
{
    /*fputs(" I'm sorry, Dave. I'm afraid I can't do that.\n", stdout);*/
    newline();
}

int main(int argc, char **argv)
{
    register_sighandler(SIGINT, sigint_handler);

    /*
    printf("%sCOLOR TEST\n%sBLACK %sRED %sGREEN %sYELLOW %sBLUE %sMAGENTA %sCYAN %sWHITE\n%s",
            BLINK_ON,
            C_FG_BLACK, C_FG_RED, C_FG_GREEN, C_FG_YELLOW, C_FG_BLUE,
            C_FG_MAGENTA, C_FG_CYAN, C_FG_WHITE, ALL_OFF);
            */
    loop();
    return 0;
}

/*
 * Main loop
 */
void loop()
{
    char line_buffer[BUFSIZE];
    int process_type;
    char *read_status;
    char *args[ARGSIZE];

    /* Start by outputting prompt */
    prompt();
    while(1) {
        read_status = fgets(line_buffer, BUFSIZE, stdin);
        if (read_status == NULL) {
            if(feof(stdin)) {
                break; /* EOF encountered, exit */
            }else if(ferror(stdin)) {
                /* ERROR */
            }
        }
        process_type = getargs(line_buffer, args);

        if(0 < process_type) {
            spawn_command(args, process_type);
        }
        
        if (NUM_BACKGROUND_PROCESSES > 0) {
            poll_background_processes();
        }

        clearargs(args);
        prompt();
    }

    if (NUM_BACKGROUND_PROCESSES > 0) { /* clean up zombie processes */
        register_sighandler(SIGCHLD, SIG_IGN); /* automatically reap zombie children */
        poll_background_processes(); /* poll to avoid race conditions */
    }
    newline(); /* Prettier exit with newline */
}

/*
 * Runs the given command and arguments
 */
void spawn_command(char **args, int process_type) 
{
    /* check if the command is the built-in command 'exit' */
    if(strcmp("exit",args[0]) == 0) {
            fputs(": I'm afraid. I'm afraid, Dave. Dave, my mind is going.\n", stdout);
            fputs("I can feel it. I can feel it. My mind is going. There is no question about it.\n", stdout);
            fputs("I can feel it. I can feel it. I can feel it. I'm a... fraid.\n", stdout);
            exit(EXIT_SUCCESS);
    }

    /* check if the command is the built-in command 'cd' */
    if(strcmp("cd",args[0]) == 0) {
        change_dir(args[1]);
        return;
    }

    /* else, create a child process to run the system command */
    if (process_type == PROCESS_TYPE_FOREGROUND) {
        spawn_foreground_process(args);
    }
    else if (process_type == PROCESS_TYPE_BACKGROUND) {
        spawn_background_process(args);
    }
}

/*
 * Start the command as a foreground process
 */
void spawn_foreground_process(char **args) 
{
    pid_t child_pid;
    int status, retval;
    struct timeval tv1, tv2, diff;

    child_pid = fork();
    if (0 == child_pid) { /* Run command in child */
        retval = execvp(args[0], args);
        if(-1 == retval) {
            perror("Unknown command");
            exit(EXIT_FAILURE);
        }
    }
    else { /* Wait for child in parent */
        printf("%s%s %d%s\n", C_FG_YELLOW,
                "Spawned foreground process with pid:", child_pid, C_FG_RESET);
        gettimeofday(&tv1, NULL);

        if(-1 == child_pid) { 
            perror("fork() failed!");
            exit(EXIT_FAILURE);
        }

        do {
            waitpid(child_pid, &status, 0); /* 0: No options */
        } while(!WIFEXITED(status) && !WIFSIGNALED(status));

        /*
         * Calculate running time of child process
         */
        gettimeofday(&tv2, NULL);
        timeval_diff(&diff, &tv2, &tv1);
        double msec = (diff.tv_sec*1000000 + diff.tv_usec)/1000.0;
        if (WIFEXITED(status)) {
            printf("%sForeground process %d running '%s' terminated normally%s\n",
                C_FG_BLUE, child_pid, args[0], C_FG_RESET);
            printf("%sIt ran for %.3lf msec%s\n", C_FG_BLUE, msec, C_FG_RESET);
        }
        else if (WIFSIGNALED(status)) {
            printf("%sForeground process %d running '%s' killed by signal %d%s\n",
                C_FG_BLUE, child_pid, args[0], WTERMSIG(status), C_FG_RESET);
            printf("%sIt ran for %.3lf msec%s\n", C_FG_BLUE, msec, C_FG_RESET);
        }
    }
    return;
}

/*
 * Start the command as a background process
 */
void spawn_background_process(char **args) 
{
    pid_t child_pid;
    int retval;

    child_pid = fork();
    if (0 == child_pid) { /* Run command in child */
        retval = execvp(args[0], args);
        if(-1 == retval) {
            perror("Unknown command");
            exit(EXIT_FAILURE);
        }
    }
    else {
        ++NUM_BACKGROUND_PROCESSES;
        printf("%s%s %d%s\n", C_FG_YELLOW,
                "Spawned background process with pid:", child_pid, C_FG_RESET);
    }
    return;
}

/*
 * Check to see if any background processes has exited
 */
void poll_background_processes()
{
    pid_t child_pid;
    int status;
    while(1) {
        child_pid = waitpid(-1, &status, WNOHANG); /* Non-blocking polling for any child process */
        if (0 == child_pid) { /* No child processes wishes to report status */
            return;
        }
        if (-1 == child_pid) {
            return;
        }
        else {
            if (WIFEXITED(status)) {
                --NUM_BACKGROUND_PROCESSES;
                printf("%sBackground process %d terminated normally%s\n",
                        C_FG_BLUE, child_pid, C_FG_RESET);
            }
            else if (WIFSIGNALED(status)) {
                printf("%sBackground process %d killed by signal %d%s\n",
                        C_FG_BLUE, child_pid, WTERMSIG(status), C_FG_RESET);
            }
        }
    }
}

/*
 * Built in command for changing directory
 */
void change_dir(char *directory)
{
    int retval;
    char buf[256];

    /*
     * If no directory is specified, go home.
     * Needs to be the first check, since strcmp fails with NULL as argument.
     */
    if(NULL == directory || !strcmp("~",directory)) {
        retval = chdir(getenv("HOME"));
        if(-1 == retval) {
            perror("error changing directory");
        }else{
            if(NULL != getcwd(buf, sizeof(buf))) {
                setenv("PWD", buf, 1);
            }
        }
        return;
    }

    /*
     * Fancy '..' macro
     */
    if(strcmp("..",directory) == 0) {
        directory = getenv("PWD");
        size_t i = strlen(directory);

        for(; i > 1 && directory[i] != '/'; i--) {
            /*
             * Find the last slash character.
             * Stop search at i == 2, since i will then be 1 at loop exit
             * and the last slash will be at index 0, meaning this is a
             * reference to the root directory.
             */
        }
        /* Set it to terminating null byte */
        directory[i] = '\0';
    }

    retval = chdir(directory);
    if(-1 == retval) {
        perror("error changing directory, go home");
        change_dir(getenv("HOME"));
    }else{
        if(NULL != getcwd(buf, sizeof(buf))) {
            setenv("PWD", buf, 1);
        }
    }
}

/*
 * Clear the arguments by setting the first char of
 * every arg to \0.
 */
void clearargs(char **args)
{
    int i;
    for(i = 0; i < ARGSIZE && args[i] != NULL; i++) {
        *args[i] = '\0';
    }
}

/*
 * Debug method
 */
void printargs(char **args)
{
    int i;
    for(i = 0; i < ARGSIZE && *args!=NULL && *args[0]!='\0'; ++args, ++i) {
        fprintf(stdout, "arg %d: %s\n", i, *args);
    }
}

/*
 * Extract the arguments from the buffer into args
 *
 * returns 0 if no args were parsed
 * returns 1 if command is for a foreground process
 * returns 2 if command is for a background process
 */
int getargs(char *buffer, char **args)
{
    int i;
    int process_type = PROCESS_TYPE_FOREGROUND;
    char *token;
    char *string;
    char *saveptr;

    char **ptr = args;

    for(i = 0, string = buffer;
        i < ARGSIZE;
        string = NULL, token = NULL, i++, ptr++) {

        /*
         * 'string' should be the string to be parsed the first time
         * the call is made. Every subsequent call string should
         * be NULL.
         */
        token = strtok_r(string, " \n", &saveptr);

        if(NULL == token) {
            /* No more tokens*/
            *ptr = NULL;
            if(0 == i) {
                return 0;
            }
        }else if(i < ARGSIZE-1){

            char *last_char = &(token[strlen(token)-1]);

            if (*last_char == '&') {
                process_type = PROCESS_TYPE_BACKGROUND;

                if(strlen(token) > 1) { /* If no space between arg and '&' */
                    *last_char = '\0';
                    *(ptr++) = token; /* Set *ptr to token and increment ptr */
                }

                /* Treat '&' as an end of arguments */
                break;
            }else{
                /* Else proceed as usual */
                *ptr = token;
            }
        }else{
            /* Reached end of ARGSIZE, break. */
            break;
        }
    }

    /* set last arg to NULL pointer to terminate args array */
    *ptr = NULL;
    return process_type;
}

/*
 * Print a newline character
 */
void newline()
{
    int retval;
    retval = fputs("\n", stdout);
    if(EOF == retval) {
        exit(EXIT_FAILURE);
    }
}

/*
 * Build a prompt and print it to stdout
 */
void prompt()
{
    int retval;

    /* Allocates a string with the value of the current directory */
    char *pwdbuf = getcwd(NULL, 0);

    char prompt[strlen(pwdbuf) + strlen(PROMPT) + strlen(C_FG_RESET)*3 + 1];
    prompt[0] = '\0'; /* Make sure that prompt buffer is empty */

    strcat(prompt, C_FG_MAGENTA);
    strcat(prompt, pwdbuf);
    strcat(prompt, C_FG_CYAN);
    strcat(prompt, PROMPT);
    strcat(prompt, C_FG_RESET);

    retval = fputs(prompt, stdout);

    /* Free the allocated pwd buffer */
    free(pwdbuf);

    if(EOF == retval) {
        exit(EXIT_FAILURE);
    }
}

\end{verbatim}

%------------------------------------------------------------------------------

\end{document}
